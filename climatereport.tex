% KOMA-Script Report-Template (A4, Times-ähnlich, deutscher Satzspiegel)
\documentclass[12pt,paper=a4,DIV=12,parskip=never,chapterprefix=false,headings=standardclasses]{scrreprt}

% --- Sprache & Encoding ---
\usepackage[T1]{fontenc}
\usepackage[utf8]{inputenc}
\usepackage[ngerman]{babel}

% --- Schrift (Times-ähnlich) & Mathematik, Mikrotypografie ---
\usepackage{newtxtext} % Textfont (Times-like, TeX Live)
\usepackage{newtxmath} % passende Matheschrift
\usepackage[final]{microtype}

% --- Absatzlayout: Einzug, kein zusätzlicher Abstand ---
\setlength{\parindent}{1.5em}
\setlength{\parskip}{0pt}

% --- Seitenstil (unten Seitenzahl, schlicht) ---
\usepackage[headsepline=false,footsepline=false]{scrlayer-scrpage}
\clearpairofpagestyles
\cfoot{\pagemark}

\usepackage{float}
% --- Hyperlinks ---
\usepackage[hidelinks]{hyperref}
\usepackage{xurl}

% --- Bilder & Captions ---
\usepackage{graphicx}
\usepackage[font=small,labelfont=bf,labelsep=period]{caption}
\usepackage{subcaption}
\setcapindent{0pt}

% --- Bequeme Referenzen (optional) ---
\usepackage[nameinlink,noabbrev]{cleveref}

\usepackage{siunitx}
\DeclareSIUnit{\ppm}{ppm}

\usepackage{amsmath}
\numberwithin{figure}{chapter}

\makeatletter
\renewcommand*{\tableofcontents}{%
  \begingroup
    \small      % Schriftgröße
    \sloppy     % vermeidet Zeilenumbrüche
    \chapter*{\contentsname}% <-- Überschrift zurück
    \@starttoc{toc}%
  \endgroup
}
\makeatother

% Kapitelname leer setzen:
\renewcaptionname{ngerman}{\chaptername}{}

% --- Titelseiten-Fonts: nicht fett ---
\setkomafont{title}{\normalfont\rmfamily\LARGE\mdseries}
\setkomafont{subtitle}{\normalfont\rmfamily\Large\mdseries}
\setkomafont{subject}{\normalfont\rmfamily\large\mdseries}
\setkomafont{author}{\normalfont\rmfamily\normalsize\mdseries}
\setkomafont{date}{\normalfont\rmfamily\normalsize\mdseries}
\setkomafont{titlehead}{\normalfont\rmfamily\mdseries}
\setkomafont{publishers}{\normalfont\rmfamily\mdseries}
\setkomafont{dedication}{\normalfont\rmfamily\mdseries}

% Kapitelüberschriften in VERSALIEN
\addtokomafont{chapter}{\MakeUppercase}

% Abschnittsüberschriften in VERSALIEN
\addtokomafont{section}{\MakeUppercase}

\begin{document}

% --- Deckblatt ---
\begin{titlepage}
\centering
\vspace*{-0.5cm}
\includegraphics[width=1.0\textwidth]{bilder/bilderKlima-0000.png}\\[1cm]

{\huge Eine kritische Überprüfung der Auswirkungen von Treibhausgasemissionen\\
auf das Klima in den Vereinigten Staaten\par}
\vfill
\begin{flushleft}
\Large
Climate Working Group\\
United States Department of Energy\\
July 23, 2025
\end{flushleft}
\vfill
\end{titlepage}

% --- Leerseite ---
\newpage
\thispagestyle{empty}
\mbox{}
\newpage

% --- Offizielle Titelseite ---
\begin{titlepage}
\centering
{\huge Eine kritische Überprüfung der Auswirkungen von Treibhausgasemissionen\\
auf das Klima in den Vereinigten Staaten\par}

\vspace{1.5cm}
\raggedright
{\Large Bericht an den US-Energieminister Christopher Wright}\\[2ex]
{\Large July 23, 2025}\\[2cm]

{\large Arbeitsgruppe Klima:}\\[0.7cm]
John Christy, Ph.D.\\
Judith Curry, Ph.D.\\
Steven Koonin, Ph.D.\\
Ross McKitrick, Ph.D.\\
Roy Spencer, Ph.D.\\[2cm]

This report is being disseminated by the Department of Energy. As such, this document was prepared
in compliance with Section 515 of the Treasury and General Government Appropriations Act for
Fiscal Year 2001 (Public Law 106-554) and information quality guidelines issued by the Department
of Energy.\\[3ex]

Copyright © 2025 United States\\[3ex]

Suggested citation:\\[1ex]

Climate Working Group (2025) A Critical Review of Impacts of Greenhouse Gas Emissions on the
U.S. Climate. Washington DC: Department of Energy, July 23, 2025
\end{titlepage}

% --- Leerseite ---
\newpage
\thispagestyle{empty}
\mbox{}
\newpage

\tableofcontents
\cleardoublepage
\pagenumbering{roman}   % I, II, III ...
\chapter*{Vorwort des Ministers}
\addcontentsline{toc}{chapter}{Vorwort des Ministers}
\section*{Energie, Integrität und die Macht des menschlichen Potentials}
Über mein Leben hinweg hatte ich das Privileg, als Energieunternehmer in einer Reihe von Bereichen zu arbeiten—Kernenergie, Geothermie, Erdgas und mehr—und ich diene jetzt als Energieminister unter Präsident Donald Trump. Aber vor allem bin ich ein Naturwissenschaftler, der moderne Energie als nichts weniger als wundersam betrachtet. Sie treibt jeden Aspekt des modernen Lebens an, befeuert jede Industrie und hat Amerika zu einer Energiemacht mit der Fähigkeit gemacht, globalen Fortschritt zu befeuern.

Der Aufstieg menschlichen Gedeihens über die letzten zwei Jahrhunderte ist eine Geschichte, die es wert ist, gefeiert zu werden. Dennoch wird uns—unaufhörlich—gesagt, dass die genau jenen Energiesysteme, die diesen Fortschritt ermöglicht haben, nun eine existenzielle Bedrohung darstellen. Kohlenwasserstoff-basierte Brennstoffe, so lautet das Argument, müssen rasch aufgegeben werden, oder wir riskieren planetaren Ruin.

Diese Sichtweise verdient Prüfung. Deshalb beauftragte ich diesen Bericht: um eine durchdachtere und wissenschaftsbasierte Unterhaltung über Klimawandel und Energie zu fördern. Mit meinem technischen Hintergrund habe ich Berichte der Zwischenstaatlichen Kommission für Klimawandel, Bewertungen der US-Regierung und die akademische Literatur überprüft. Ich habe mich auch mit vielen Klimawissenschaftlern auseinandergesetzt, einschließlich der Autoren dieses Berichts.

Was ich festgestellt habe, ist, dass Medienberichterstattung oft die Wissenschaft verzerrt. Viele Menschen kommen mit einer Sicht auf Klimawandel davon, die übertrieben oder unvollständig ist. Um Klarheit und Ausgewogenheit zu schaffen, bat ich ein vielfältiges Team unabhängiger Experten, den aktuellen Stand der Klimawissenschaft kritisch zu überprüfen, mit einem Fokus darauf, wie er sich auf die Vereinigten Staaten bezieht.

Ich wählte diese Autoren nicht aus, weil wir uns immer einig sind—weit gefehlt. Tatsächlich stimmen sie möglicherweise nicht immer miteinander überein. Aber ich wählte sie wegen ihrer Gründlichkeit, Ehrlichkeit und Bereitschaft, die Debatte zu erheben. Ich übte keine Kontrolle über ihre Schlussfolgerungen aus. Was Sie lesen werden, sind ihre Worte, gezogen aus den besten verfügbaren Daten und wissenschaftlichen Bewertungen.

Ich habe den Bericht sorgfältig überprüft, und ich glaube, er repräsentiert den Stand der Klimawissenschaft heute getreu. Dennoch mögen viele Leser von seinen Schlussfolgerungen überrascht sein—die sich in wichtigen Weisen von der Mainstream-Erzählung unterscheiden. Das ist ein Zeichen dafür, wie weit die öffentliche Unterhaltung von der Wissenschaft selbst abgedriftet ist.

Um wieder auf Kurs zu kommen, brauchen wir offene, respektvolle und informierte Debatte. Deshalb lade ich zu öffentlichen Kommentaren zu diesem Bericht ein. Ehrliche Prüfung und wissenschaftliche Transparenz sollten im Herzen unserer Politikgestaltung stehen.

Klimawandel ist real, und er verdient Aufmerksamkeit. Aber er ist nicht die größte Bedrohung für die Menschheit. Diese Auszeichnung gehört der globalen Energiearmut. Als jemand, der Daten schätzt, weiß ich, dass die Verbesserung der menschlichen Lage davon abhängt, den Zugang zu zuverlässiger, erschwinglicher Energie zu erweitern. Klimawandel ist eine Herausforderung—keine Katastrophe. Aber fehlgeleitete Politik, die auf Angst statt auf Fakten basiert, könnte wirklich das menschliche Wohlergehen gefährden.

Wir stehen an der Schwelle einer neuen Ära der Energieführerschaft. Wenn wir Innovation befähigen, anstatt sie zu beschränken, kann Amerika die Welt dabei anführen, sauberere, reichlichere Energie bereitzustellen—Milliarden aus der Armut zu heben, unsere Wirtschaft zu stärken und nebenbei unsere Umwelt zu verbessern.

\cleardoublepage
\chapter*{Kurzfassung}
\addcontentsline{toc}{chapter}{Kurzfassung}
Dieser Bericht überprüft wissenschaftliche Gewissheiten und Ungewissheiten darin, wie anthropogenes Kohlendioxid (CO$_2$) und andere Treibhausgasemissionen das Klima, extreme Wetterereignisse und ausgewählte Metriken des gesellschaftlichen Wohlergehens der Nation beeinflusst haben oder beeinflussen werden. Diese Emissionen erhöhen die Konzentration von CO$_2$ in der Atmosphäre durch einen komplexen und variablen Kohlenstoffkreislauf, wobei ein Teil des zusätzlichen CO$_2$ für Jahrhunderte in der Atmosphäre verbleibt.

Erhöhte Konzentrationen von CO$_2$ verstärken direkt das Pflanzenwachstum, tragen global zur "Ergrünung" des Planeten bei und erhöhen die landwirtschaftliche Produktivität [Abschnitt 2.1, Kapitel 9]. Sie machen auch die Ozeane weniger alkalisch (senken den pH-Wert). Das ist möglicherweise schädlich für Korallenriffe, obwohl die jüngste Erholung des Great Barrier Reef anderes nahelegt [Abschnitt 2.2].

Kohlendioxid wirkt auch als Treibhausgas und übt einen erwärmenden Einfluss auf Klima und Wetter aus [Abschnitt 3.1]. Klimawandelprojektionen erfordern Szenarien zukünftiger Emissionen. Es gibt Beweise, dass Szenarien, die in der Auswirkungsliteratur weit verbreitet sind, beobachtete und wahrscheinliche zukünftige Emissionstrends überschätzt haben [Abschnitt 3.1].

Die mehreren Dutzend globalen Klimamodelle der Welt bieten wenig Orientierung darüber, wie sehr das Klima auf erhöhtes CO$_2$ reagiert, wobei die durchschnittliche Oberflächenerwärmung unter einer Verdopplung der CO$_2$-Konzentration von \SI{1.8}{\celsius} bis \SI{5.7}{\celsius} reicht [Abschnitt 4.2]. Datengesteuerte Methoden ergeben einen niedrigeren und engeren Bereich [Abschnitt 4.3]. Globale Klimamodelle laufen im Allgemeinen \emph{heiß} in ihrer Beschreibung des Klimas der letzten Jahrzehnte -- zu viel Erwärmung an der Oberfläche und zu viel Verstärkung der Erwärmung in der unteren und mittleren Troposphäre [Abschnitte 5.2-5.4]. Die Kombination übermäßig empfindlicher Modelle und unplausiblen extremen Szenarien für zukünftige Emissionen ergibt übertriebene Projektionen zukünftiger Erwärmung.

Die meisten extremen Wetterereignisse in den USA zeigen keine langfristigen Trends. Behauptungen erhöhter Häufigkeit oder Intensität von Hurrikanen, Tornados, Überschwemmungen und Dürren werden nicht durch historische US-Daten gestützt [Abschnitte 6.1-6.7]. Zusätzlich werden Waldmanagementpraktiken oft übersehen bei der Bewertung von Veränderungen in der Waldbrandaktivität [Abschnitt 6.8]. Der globale Meeresspiegel ist seit 1900 um etwa 8 Zoll gestiegen, aber es gibt signifikante regionale Variationen, die primär durch lokale Landabsenkung angetrieben werden; US-Pegelmessungen zeigen insgesamt keine offensichtliche Beschleunigung im Meeresspiegelanstieg über die historische Durchschnittsrate hinaus [Kapitel 7].

Die Zuordnung von Klimawandel oder extremen Wetterereignissen zu menschlichen CO$_2$-Emissionen wird durch natürliche Klimavariabilität, Datenbeschränkungen und inhärente Modellmängel herausgefordert [Kapitel 8]. Darüber hinaus könnte der Beitrag der Sonnenaktivität zur Erwärmung des späten 20. Jahrhunderts unterschätzt sein [Abschnitt 8.3.1].

Sowohl Modelle als auch Erfahrung legen nahe, dass CO$_2$-induzierte Erwärmung wirtschaftlich weniger schädlich sein könnte als gemeinhin geglaubt, und übermäßig aggressive Minderungspolitiken könnten sich als schädlicher erweisen denn als vorteilhaft [Kapitel 9, 10, Abschnitt 11.1]. Schätzungen der gesellschaftlichen Kosten von Kohlenstoff, die versuchen, den wirtschaftlichen Schaden von CO$_2$-Emissionen zu quantifizieren, sind sehr empfindlich gegenüber ihren zugrundeliegenden Annahmen und liefern daher begrenzte unabhängige Informationen [Abschnitt 11.2].

Von US-Politikaktionen wird erwartet, dass sie undetektierbar kleine direkte Auswirkungen auf das globale Klima haben, und alle Effekte werden nur mit langen Verzögerungen auftreten [Kapitel 12].

\cleardoublepage
\chapter*{Vorwort}
\addcontentsline{toc}{chapter}{Vorwort}
Dieses Dokument entstand Ende März 2025, als Minister Wright eine unabhängige Gruppe zusammenstellte, um einen Bericht über Themen in der Klimawissenschaft zu schreiben, die für die Energiepolitikgestaltung relevant sind, einschließlich Beweisen und Perspektiven, die den Mainstream-Konsens herausfordern. Wir stimmten zu, die Arbeit unter der Bedingung zu übernehmen, dass es keine redaktionelle Aufsicht durch den Minister, das Energieministerium oder anderes Regierungspersonal geben würde. Diese Bedingung wurde während des gesamten Prozesses eingehalten, und das Schreibteam hat mit voller Unabhängigkeit gearbeitet.

Die Gruppe begann Anfang April mit der Arbeit mit einer Frist am 28. Mai, einen Entwurf zur internen DOE-Überprüfung zu liefern. Die kurze Zeitlinie und die technische Natur des Materials bedeuteten, dass wir nicht alle Themen umfassend überprüfen konnten. Vielmehr wählten wir aus, uns auf Themen zu konzentrieren, die von einer ernsthaften, etablierten akademischen Literatur behandelt werden; die für unseren Auftrag relevant sind; die in jüngsten Bewertungsberichten heruntergespielt oder abwesend sind; und die innerhalb unserer Kompetenz liegen.

Während der Bericht für Nicht-Experten zugänglich sein soll, haben wir einiges einführende oder erklärende Material weggelassen, das leicht anderswo zugänglich ist. Wir haben auch nicht versucht, die gesamte Literatur zu den behandelten Themen zu überblicken. Wir haben uns so weit wie möglich auf Literatur konzentriert, die seit 2020 veröffentlicht wurde, und frühere IPCC- und NCA-Bewertungsberichte referenziert. Wir haben auch Daten bis 2024 verwendet, wo möglich.

Das Schreibteam ist Minister Wright dankbar für die Gelegenheit, diesen Bericht zu erstellen, und für seine Unterstützung unabhängiger wissenschaftlicher Bewertung und offener wissenschaftlicher Debatte. Wir sind auch einem Team anonymer DOE- und nationaler Labor-Gutachter dankbar, deren Input geholfen hat, den finalen Bericht zu verbessern.
\vspace{2ex}

\noindent
John Christy, Ph.D.\\[1em]
Judith Curry, Ph.D.\\[1em]
Steven Koonin, Ph.D.\\[1em]
Ross McKitrick, Ph.D.\\[1em]
Roy Spencer, Ph.D.

\cleardoublepage
\chapter*{TEIL I: DIREKTER MENSCHLICHER EINFLUSS AUF ÖKOSYSTEME UND DAS KLIMA}
\addcontentsline{toc}{chapter}{TEIL I: DIREKTER MENSCHLICHER EINFLUSS AUF ÖKOSYSTEME UND DAS KLIMA}
\cleardoublepage
\pagenumbering{arabic}
\chapter{Kohlendioxid als Schadstoff}
\paragraph{Kapitelzusammenfassung}
\begin{quote}
Kohlendioxid (CO$_2$) unterscheidet sich in vielerlei Hinsicht von den sogenannten Criteria Air Pollutants. Es beeinflusst nicht die lokale Luftqualität und hat keine menschlichen toxikologischen Implikationen bei Umgebungskonzentrationen. Es ist ein Anliegen wegen seiner Auswirkungen auf das globale Klima.
\end{quote}

Der \emph{Clean Air Act} von 1970 definierte sechs sogenannte \emph{Criteria Air Contaminants}, die der Regulierung unterliegen (EPA): Feinstaub, bodennahes Ozon, Schwefeldioxid, Stickstoffdioxid, Blei und Kohlenmonoxid. Im Jahr 2007 entschied der Oberste Gerichtshof, dass Treibhausgase (CO$_2$ unter ihnen) auch "Schadstoffe" seien, die der Regulierung unter dem \emph{Clean Air Act} unterliegen (Mass. v. EPA, 2007).

Während die Definition von \emph{Schadstoff} letztendlich eine rechtliche Angelegenheit ist, gibt es wichtige wissenschaftliche Unterscheidungen zwischen CO$_2$ und den \emph{Criteria Air Contaminants}. Letztere unterliegen der regulatorischen Kontrolle, weil sie lokale Probleme verursachen, die von Konzentrationen abhängen, einschließlich Belästigungen (Geruch, Sichtbarkeit), Schäden an Pflanzen und, bei ausreichend hohen Expositionslevels, toxikologischen Effekten bei Menschen. Im Gegensatz dazu ist CO$_2$ geruchlos, beeinflusst nicht die Sichtbarkeit und hat keine toxikologischen Effekte bei Umgebungskonzentrationen. Es ist ein natürlich vorkommender Teil der Atmosphäre und eine Schlüsselkomponente der menschlichen und pflanzlichen Atmung. CO$_2$ ist essentiell für die pflanzliche Photosynthese, und höhere Konzentrationen sind vorteilhaft für die Vegetation. In diesen Aspekten ist CO$_2$ dem Wasserdampf ähnlich.

Die heutige Umgebungsluft im Freien enthält etwa 430 Teile pro Million (ppm) CO$_2$, steigend um etwa 2 ppm pro Jahr. Die U.S. Occupational Safety and Health Administration gibt Richtlinien für Arbeitsplätze in Innenräumen heraus, in denen erhöhtes CO$_2$ auftreten könnte, wie etwa dort, wo Trockeneis verwendet wird. Das zulässige Expositionslimit beträgt 5.000 ppm über 8 Stunden (OSHA, 2024). Allen et al. (2015) berichteten über Beweise für verminderte Leistung bei einigen kognitiven Aufgaben unter Arbeitern in Bürokabinen, wenn sie CO$_2$-Konzentrationen über 1.000-1.500 ppm ausgesetzt waren. Diese Werte sind weit größer als jeder plausible Umgebungswert im Freien bis zum Ende des 22. Jahrhunderts.

Die wachsende Menge an CO$_2$ in der Atmosphäre beeinflusst das Erdsystem direkt, indem sie das Pflanzenwachstum fördert (globale Ergrünung), dadurch landwirtschaftliche Erträge steigert und die Alkalinität der Ozeane neutralisiert. Aber die primäre Sorge bezüglich CO$_2$ ist seine Rolle als Treibhausgas (GHG), das die Energiebilanz der Erde verändert und den Planeten erwärmt. Wie das Klima auf diesen Einfluss reagieren wird, ist eine komplexe Frage, die einen Großteil dieses Berichts beschäftigen wird.

\vfill
Literaturverzeichnis:

Allen, J., Macnaughton, P., Satish, U., et al. (2015). Associations of cognitive function scores with carbon dioxide, ventilation, and volatile organic compound exposures in office workers: A controlled exposure study of green and conventional office environments. Environmental Health Perspectives. 124. https://doi.org/10.1289/ehp.1510037

Massachusetts v. Environmental Protection Agency, 549 U.S. 497 (2007).\\ https://www.oyez.org/cases/2006/05-1120

U.S. Environmental Protection Agency. (n.d.). Criteria air pollutants.\\ https://www.epa.gov/criteria-air-pollutants

U.S. Occupational Safety and Health Administration. (2024). OSHA occupational chemical database: Carbon dioxide. https://www.osha.gov/chemicaldata/183

\chapter{Direkte Auswirkungen von CO$_2$ auf die Umwelt}
\paragraph{Kapitelzusammenfassung}
\begin{quote}
CO$_2$ verstärkt die Photosynthese und verbessert die Wassernutzungseffizienz von Pflanzen und fördert dadurch das Pflanzenwachstum. Globale Ergrünung aufgrund teilweise erhöhter CO$_2$-Konzentrationen in der Atmosphäre ist auf allen Kontinenten gut etabliert.

CO$_2$-Absorption im Meerwasser macht die Ozeane weniger alkalisch. Der jüngste pH-Rückgang liegt innerhalb der Bandbreite natürlicher Variabilität auf jahrtausendlangen Zeitskalen. Das meiste Meeresleben entwickelte sich, als die Ozeane leicht sauer waren. Abnehmender pH-Wert könnte Korallen negativ beeinflussen, obwohl das australische Great Barrier Reef in den letzten Jahren beträchtliches Wachstum gezeigt hat.
\end{quote}

\section{CO$_2$ als Beitrag zur globalen Ergrünung}

Die wachsende CO$_2$-Konzentration in der Atmosphäre hat den wichtigen positiven Effekt, das Pflanzenwachstum zu fördern, indem sie die Photosynthese verstärkt und die Wassernutzungseffizienz verbessert. Das ist im unten diskutierten \emph{globalen Ergrünungs}-Phänomen evident sowie in den verbesserten landwirtschaftlichen Erträgen, die in Kapitel 10 diskutiert werden. Hier konzentrieren wir uns nur auf CO$_2$-Düngung; Forschung zu kombinierten Effekten durch Temperatur- und Niederschlagsveränderungen wird in Kapitel 10 diskutiert.

\subsection{Messung der globalen Ergrünung}

\emph{Ergrünung} bezieht sich auf eine Zunahme des Anteils der Erdoberfläche, der von Pflanzen bedeckt ist. Sie kann durch den \emph{Blattflächenindex} (LAI) quantifiziert werden, der per Satellit gemessen wird. Viele Studien über das letzte Jahrzehnt haben ein globales Ergrünungsmuster (Zunahme des LAI) bestätigt, das teilweise auf steigende CO$_2$-Konzentrationen zurückzuführen ist. Zhu et al. (2016) war eine der ersten Studien, die berichtete, dass globale Ergrünung mit Satellitensensoren detektierbar war. Von 1982 bis 2011 entdeckten sie Ergrünung über 25-50 Prozent der Erde versus \emph{Bräunung} über nur vier Prozent und schrieben 70 Prozent der Ergrünung steigenden CO$_2$-Konzentrationen zu (siehe Abbildung 2.1). Andere Beiträge umfassten Landnutzungsänderungen, Erwärmung und Stickstoff. Der CO$_2$ zuschreibbare Anteil war in den Tropen am größten; andere Faktoren spielten dominantere Rollen in CONUS.

Zeng et al. (2017) bestätigten das Ergrünungsmuster und bemerkten, dass es über dreißig Jahre 8 Prozent zur globalen Blattfläche hinzugefügt hatte und dass Ergrünung die Erwärmung milderte. Ergrünung wurde global beobachtet. Chen et al. (2019) zeigen, dass in China und Indien viel davon durch Landmanagement-Änderungen angetrieben wird. So macht China nur 6,6 Prozent der globalen bewachsenen Fläche aus, aber 25 Prozent der globalen Netto-Zunahme des LAI. Piao et al. (2020) bemerkten, dass Ergrünung sogar in der Arktis beobachtbar war. CO$_2$-Düngungseffekte werden von lokaler Temperatur und Nährstoff- und Wasserverfügbarkeit beeinflusst, die alle regional variieren.

Während Pflanzenmodelle eine erhöhte Photosynthese als Antwort auf steigendes CO$_2$ vorhersagen, berichteten Haverd et al. (2020) eine CO$_2$-Düngungsrate, die viel größer war als Modellvorhersagen. Das heißt, CO$_2$-Düngung hatte einen Anstieg der beobachteten globalen Photosynthese um 30 Prozent seit 1900 angetrieben, versus 17 Prozent, die von Pflanzenmodellen vorhergesagt wurden. Falls zutreffend, würde dies andeuten, dass globale Modelle der sozioökonomischen Auswirkungen steigenden CO$_2$ die Vorteile für Feldfrüchte und Landwirtschaft unterschätzt haben. Keenan et al. (2023) schätzten jedoch eine niedrigere Düngungsrate, die mehr mit Modellen übereinstimmte. Die Verbindung zwischen CO$_2$-Düngung und Landwirtschaft wird in Kapitel 9 diskutiert.

\begin{figure}[H]
\begin{center}
\includegraphics[width=1.0\textwidth]{bilder/bilderKlima-0002.jpg}\\[1cm]
\end{center}
\caption{Trends beim durchschnittlichen Blattflächenindex (LAI). Quelle: Zhu et al. 2016 Abbildung 3.}
\end{figure}

Piao et al. (2020) und Chen et al. (2024) berichten, dass der Ergrünungstrend ohne Anzeichen einer Verlangsamung fortsetzt, und CO$_2$-Düngung bleibt der dominante Treiber.

\subsection{Photosynthese und CO$_2$-Konzentrationen}

Pflanzen bauen Biomasse durch Photosynthese auf, einen Prozess, der Kohlendioxid, Wasser und Licht in Zucker umwandelt. Das für die Photosynthese verantwortliche Pflanzenenzym ist Ribulose-1,5-bisphosphat-carboxylase/oxygenase oder \emph{Rubisco}. Photosynthese wird eingeleitet, wenn CO$_2$ an der Oberfläche des Rubisco-Enzyms verfügbar ist, wo es in ein Molekül mit 3 Kohlenstoffatomen umgewandelt und danach in die Pflanzenmasse eingebaut wird. Dies wird als \emph{C3}-Prozess bezeichnet.

Rubisco wird geschätzt, vor etwa 3 Milliarden Jahren entstanden zu sein. Über geologische Zeit waren die atmosphärischen CO$_2$-Konzentrationen der Erde normalerweise viele Male höher als heute. Vor etwa 400 Millionen Jahren lagen CO$_2$-Konzentrationen schätzungsweise bei 2.000-4.000 ppm und waren für den größten Teil des Zeitraums von 200 bis 50 Millionen Jahren bei oder über 1.000 ppm (Berner 2006, Judd et al. 2024). Über die letzten 35 Millionen Jahre ist die atmosphärische CO$_2$-Konzentration stetig gefallen und fiel auf bis zu 170 ppm während Vereisung (Gerhart und Ward 2010). Während die moderne Änderungsrate von CO$_2$ im Vergleich zu früheren Zeiträumen hoch sein mag, zeigen die geologischen Beweise, dass Pflanzen und Tiere sich unter viel höheren CO$_2$-Konzentrationen als gegenwärtig entwickelten.

Als Antwort auf niedrige CO$_2$-Bedingungen entwickelten einige Pflanzen einen anderen photosynthetischen Weg namens \emph{C4}, bei dem CO$_2$ in der Nähe von Rubisco konzentriert wird, wodurch der \emph{C3}-Prozess effizienter funktionieren kann. Für landwirtschaftliche Zwecke sind die Pflanzenkategorien:
\begin{itemize}
\item C3: Reis, Weizen, Sojabohnen und die meisten anderen Feldfrüchte
\item C4: Mais, Zuckerrohr, Hirse, Sorghum	
\end{itemize}

Hätten atmosphärische CO$_2$-Konzentrationen weiter abgenommen, wäre das Pflanzenwachstum zurückgegangen und schließlich aufgehört. Unter \SI{180}{ppm} sind die Wachstumsraten vieler C3-Arten um 40-60 Prozent im Vergleich zu \SI{350}{ppm} reduziert (Gerhart und Ward 2010), und das Wachstum ist unter experimentellen Bedingungen von \SIrange{60}{140}{ppm} CO$_2$ vollständig zum Stillstand gekommen. Einige C4-Pflanzen können noch bei Konzentrationen von nur \SI{10}{ppm} wachsen, wenn auch sehr langsam (Gerhart und Ward 2010).


\begin{figure}[H]
\begin{center}
\includegraphics[width=0.6\textwidth]{bilder/bilderKlima-0003.jpg}\\[1cm]
\end{center}
\caption{Wachstum von \emph{Abutilon theophrasti} nach 14 Tagen unter identischen Bedingungen, jedoch mit den
angegebenen Schwankungen des CO2-Gehalts. Quelle: Gerhart und Ward (2010). Hinweis: „Aktuell” entspricht
1988 im Bild.}
\end{figure}

Die aktuellen CO$_2$-Konzentrationen liegen bei etwa \SI{430}{ppm}, gegenüber \SI{280}{ppm} in den frühen 1800er Jahren. Die positive Reaktion von Pflanzen auf zusätzliches CO$_2$ ist in Abbildung 2.2 dargestellt, reproduziert aus Gerhart und Ward (2010). Sie zeigt den Wachstumseffekt von CO$_2$ auf Samtpappel (Abutilon theophrasti) Sämlinge über 14 Tage unter kontrollierten Bedingungen, wo nur die CO$_2$-Exposition variiert wird. Die Zuwächse durch Erhöhung von CO$_2$ von \SI{150}{ppm} auf \SI{350}{ppm} setzen sich bei einer weiteren Verdopplung auf \SI{700}{ppm} fort.

Über die letzten 60+ Jahre gab es Tausende von Studien über die Reaktion von Pflanzen auf steigende CO$_2$-Konzentrationen. Das überwältigende Thema ist, dass Pflanzen, besonders C3-Pflanzen, von zusätzlichem CO$_2$ profitieren. Es gibt zwei Mechanismen, durch die CO$_2$ einen Wachstumsvorteil verleiht:

\begin{itemize}
\item Verstärkte Photosynthese über die oben beschriebenen Stoffwechselwege.
\item Erhöhte Wassernutzungseffizienz. Dies entsteht, weil Pflanzen CO$_2$ aufnehmen, indem sie die Stomata (Poren) auf der Blattoberfläche öffnen. Wenn CO$_2$ knapp ist, müssen die Stomata für lange Zeiträume weit offen gehalten werden, wodurch Wasser verdunsten kann. Unter angereicherten CO$_2$-Bedingungen bleiben die Stomata für längere Zeiträume geschlossen, wodurch die Pflanze Wasser länger zurückhalten kann und so die Wassernutzungseffizienz erhöht wird.
\end{itemize}

Spezifische Auswirkungen des Klimawandels auf die US-Landwirtschaft werden in Kapitel 9 überprüft.

\subsection{Steigendes CO$_2$ und Wassernutzungseffizienz von Feldfrüchten}

Deryng et al. (2016) untersuchten Beweise zur Wasserproduktivität von Feldfrüchten (CWP), dem Ertrag pro verwendete Wassereinheit, und lenkten Aufmerksamkeit auf das Potenzial von CO$_2$, sowohl die Photosynthese zu verstärken als auch die Transpiration auf Blattebene (Wasserverlust während der Blattatmung) zu reduzieren. Sie untersuchten alle verfügbaren FACE-Daten (Free Air CO$_2$ Enrichment—siehe Kapitel 9) zu Feldfruchtertragsänderungen für Mais, Weizen, Reis und Sojabohnen und kombinierten sie mit Feldfruchtemodell-Daten, die Ertragsreaktionen bis 2080 unter dem extremen RCP8.5-Emissionsszenario in vier Anbauregionen (Tropen, Trocken, Gemäßigt und Kalt) simulierten, von denen jede in regenabhängige und bewässerte Unterregionen aufgeteilt war. Sie berichteten, dass Modelle ohne CO$_2$-Düngung CWP-Verluste in jeder Region vorhersagten, aber diese wurden durch CO$_2$-Düngung mehr als ausgeglichen, sodass alle Regionen einen Netto-CWP-Gewinn zeigten.

Deryng et al. (2016) berichteten auch, dass negative Auswirkungen der Erwärmung auf Weizen- und Sojabohnenerträge vollständig durch CWP-Gewinne ausgeglichen und um bis zu 90 Prozent für Reis und 60 Prozent für Mais gemildert wurden.

Ähnlich bemerkten Cheng et al. (2017), dass die erhöhte Bruttoprimärproduktion von 1982 bis 2011 aufgrund steigender CO$_2$-Aufnahme von so großen CWP-Gewinnen begleitet war, dass der globale Wasserverbrauch von Pflanzen nicht gestiegen war, trotz der zusätzlichen Biomasse.

Deryng et al. (2016) nahmen an, dass Klimawandel \emph{Wasserknappheit verschärfen} würde. Doch während Modelle vorhersagen, dass Trockengebiete sich unter Klimaerwärmung ausdehnen werden, zeigen aktuelle Daten das Gegenteil: Ergrünung geschieht sogar in trockenen Gebieten. Zhang et al. (2024) berichten, dass aufgrund erhöhter CO$_2$-Konzentrationen \emph{zunehmende Trockenheit in Trockengebieten nicht zu einem allgemeinen Verlust der Vegetationsproduktivität führen wird}; höchstens 4 Prozent der derzeit trockenen Gebiete werden erhöhte Wüstenbildung erleben.

\subsection{CO$_2$-Düngungsvorteile in IPCC-Berichten}

Das IPCC hat globale Ergrünung und CO$_2$-Düngung landwirtschaftlicher Feldfrüchte nur minimal diskutiert. Das Thema wird kurz an einigen Stellen im Hauptteil des 6. und früherer IPCC-Bewertungsberichte anerkannt, aber in allen Zusammenfassungsdokumenten weggelassen. Abschnitt 2.3.4.3.3 des AR6 Working Group I-Berichts, betitelt \emph{globale Ergrünung und Bräunung}, weist darauf hin, dass der IPCC-Sonderbericht über Klimawandel und Land mit hoher Zuversicht geschlossen hatte, dass Ergrünung global über die letzten 2-3 Jahrzehnte zugenommen hatte. Er diskutiert dann, dass es Variationen im Ergrünungstrend zwischen Datensätzen gibt, und schließt, dass sie zwar hohe Zuversicht haben, dass Ergrünung aufgetreten ist, aber niedrige Zuversicht in das Ausmaß des Trends. Es gibt auch kurze Erwähnungen von CO$_2$-Düngungseffekten und Verbesserungen der Wassernutzungseffizienz in einigen anderen Kapiteln der AR6 Working Groups I und II-Berichte.

Insgesamt diskutieren jedoch die Policymaker Summaries, Technical Summaries und Synthesis Reports von AR5 und AR6 das Thema nicht.


\section{Die alkalischen Ozeane}

\subsection{Verändernder pH-Wert}

Eine neutrale wässrige Lösung hat einen pH-Wert von 7,0, während eine mit pH größer als 7,0 alkalisch (auch basisch genannt) und mit pH kleiner als 7,0 sauer ist. Der heutige globale Durchschnitts-pH-Wert von Oberflächenmeerwasser wird auf 8,04 geschätzt (\emph{Copernicus Marine Service 2025}, Abbildung 2.3), gegenüber einem geschätzten Wert von 8,2 in vorindustrieller Zeit (Gattuso und Hansson, 2011). Als CO$_2$-Konzentrationen in der Atmosphäre stiegen, absorbierten die Ozeane mehr, was ihren pH-Wert senkt. Abhängig von der Pufferkapazität der Ozeane wird erwartet, dass sie mit der Zeit etwas weniger alkalisch werden, was mit dem beobachteten pH-Rückgang übereinstimmt.

\begin{figure}[H]
\begin{center}
\includegraphics[width=0.8\textwidth]{bilder/bilderKlima-0004.jpg}\\[1cm]
\end{center}
\caption{pH-Wert der Ozeane 1985–2022. Quelle: Copernicus Marine Service 2025}
\end{figure}

Während dieser Prozess oft \emph{Ozeanversauerung} genannt wird, ist das eine Fehlbezeichnung, weil nicht erwartet wird, dass die Ozeane sauer werden; \emph{Ozeanneutralisierung} wäre genauer. Selbst wenn das Wasser sauer würde, wird angenommen, dass sich Leben in den Ozeanen entwickelte, als die Ozeane leicht sauer mit pH 6,5 bis 7,0 waren (Krissansen-Totton et al., 2018). Auf der Zeitskala von Tausenden von Jahren zeigen Bor-Isotop-Proxy-Messungen, dass der Ozean-pH-Wert während der letzten Vereisung (bis vor etwa 20.000 Jahren) um 7,4 oder 7,5 lag und auf heutige Werte anstieg, als sich die Welt während der Enteisung erwärmte (Rae et al., 2018). Somit scheinen Ozeanbiota widerstandsfähig gegenüber natürlichen langfristigen Veränderungen des Ozean-pH-Werts zu sein, da Meeresorganismen einer weiten Bandbreite von pH-Werten ausgesetzt waren.

\subsection{Korallenriff-Veränderungen}

Es gibt Bedenken, dass ein abnehmender pH-Wert des Meerwassers die Verkalkungsrate von Korallenriffen reduzieren wird. Aber Korallenriffe ertragen bereits große pH-Schwankungen, teilweise aufgrund täglicher photosynthetischer Aktivität im Riff; gemessene pH-Werte reichen von 9,4 während des Tages bis 7,5 nachts (Revelle und Fairbridge, 1957). De'ath et al. (2009) berichteten, dass Verkalkungsraten am Great Barrier Reef von 1990 bis 2005 um 14,2 Prozent zurückgegangen waren. Ridd et al. (2013) verwendeten jedoch andere Methoden und fanden keinen signifikanten Trend in den Verkalkungsraten.

Jüngste Daten vom Australian Institute of Marine Science zeigen, dass sich das Great Barrier Reef stark erholt hat. Der Korallenbedeckungsgrad erreichte 2022 36-Jahres-Höchststände in zwei Dritteln des Riffs. Das Institute kommentierte: \emph{Das war das dritte Jahr in Folge mit weit verbreiteter Erholung} (Australian Institute of Marine Science, 2022). Die \emph{Woods Hole Oceanographic Institution} bemerkte 2023: \emph{Das Great Barrier Reef macht ein bemerkenswertes Comeback} (Woods Hole Oceanographic Institution, 2023).

\begin{figure}[H]
\begin{center}
\includegraphics[width=1.0\textwidth]{bilder/bilderKlima-0005.png}\\[1cm]
\end{center}
\caption{Hartkorallenbedeckung in drei Regionen des \emph{Great Barrier Reef} von 1985 bis 2023. Quelle: AIMS 2023.}
\end{figure}

Es scheint, dass Korallen anpassungsfähiger sind als früher gedacht. Die Vorfahren moderner Korallen erschienen vor etwa 245 Millionen Jahren. CO$_2$-Konzentrationen waren für mehr als 200 Millionen Jahre danach viele Male höher als heute. Ein Großteil der öffentlichen Diskussion über die Auswirkungen der Ozean-"Versauerung" auf Meeresbiota war einseitig und übertrieben.

Ähnlich fand eine Meta-Analyse (Clements et al., 2021) der negativen Auswirkungen der Ozeanversauerung auf das Verhalten von Rifffischen das, was sie einen \emph{Decline-Effekt} nannten: anfänglich dramatische Schlussfolgerungen, die in prominenten Zeitschriften veröffentlicht wurden und scheinbar große Auswirkungen der Versauerung zeigten, tendierten dazu, von nachfolgenden Studien mit größeren Stichprobengrößen gefolgt zu werden, die viel kleinere und typischerweise nicht-existente Effekte ergaben. Sie fordern ihre Kollegen auf, Forschungspraktiken zu verbessern, um dem \emph{Decline-Effekt} entgegenzuwirken:

Die überwiegende Mehrheit der Studien mit großen Effektgrößen in diesem Bereich tendiert dazu, durch niedrige Stichprobengrößen charakterisiert zu sein, wird aber dennoch in hochrangigen Zeitschriften veröffentlicht und hat einen unverhältnismäßigen Einfluss auf das Feld in Bezug auf Zitationen. Wir behaupten, dass Ozeanversauerung einen vernachlässigbaren direkten Einfluss auf Fischverhalten hat, und wir setzen uns für verbesserte Ansätze ein, um das Potenzial für einen Decline-Effekt in zukünftigen Forschungsrichtungen zu minimieren (Clements et al., 2021).

Zusammenfassend ist Meeresleben komplex, und vieles davon entwickelte sich, als die Ozeane relativ zur Gegenwart sauer waren. Die Vorfahren moderner Korallen erschienen vor etwa 245 Millionen Jahren. CO$_2$-Konzentrationen waren für mehr als 200 Millionen Jahre danach viele Male höher als heute. Ein Großteil der öffentlichen Diskussion über die Auswirkungen der Ozean-"Versauerung" auf Meeresbiota war einseitig und übertrieben.

\vfill
Literaturverzeichnis:

AR6: Intergovernmental Panel on Climate Change Sixth Assessment Report (2021) Working Group I Contribution. www.ipcc.ch.

Australian Institute of Marine Science. (2022). Continued coral recovery leads to 36-year highs across two-thirds of the Great Barrier Reef. \url{https://www.aims.gov.au/sites/default/files/2022-08/AIMS_LTMP_Report_on\%20GBR_coral_status_2021_2022_040822F3.pdf}

Beeden, R., Maynard, J., Puotinen, M., Marshall, P., Dryden, J., Goldberg, J., and Williams, G. (2015). Impacts and recovery from Severe Tropical Cyclone Yasi on the Great Barrier Reef. PLOS ONE, 10, e0121272. https://doi.org/10.1371/journal.pone.0121272

Berner, R. A. (2006). GEOCARBSULF: A combined model for Phanerozoic atmospheric O$_2$ and CO$_2$. Geochimica et Cosmochimica Acta, 70, 5653–5664.

Browman, H. I. (2016). Applying organized scepticism to ocean acidification research. ICES Journal of Marine Science, 73(3), 529.1–536. https://doi.org/10.1093/icesjms/fsw010

Chen, C., Park, T., Wang, X., Piao, S., Xu, B., Chaturvedi, R. K., and Myneni, R. B. (2019). China and India lead in greening of the world through land-use management. Nature Sustainability, 2, 122–129. https://www.nature.com/articles/s41893-019-0220-7

Chen, X., Wang, Y., Liu, Y., and Piao, S. (2024). The global greening continues despite increased drought stress since 2000. Global Ecology and Conservation, 49, e02791. https://www.sciencedirect.com/science/article/pii/S2351989423004262

Cheng, L., Zhang, L., Wang, Y. P., et al. (2017). Recent increases in terrestrial carbon uptake at little cost to the water cycle. Nature Communications, 8, 110. https://doi.org/10.1038/s41467-017-00114-5

Clements, J. C., Sundin, J., Clark, T. D., and Jutfelt, F. (2022). Meta-analysis reveals an extreme "decline effect" in the impacts of ocean acidification on fish behavior. PLOS Biology, 20(2), e3001511. https://doi.org/10.1371/journal.pbio.3001511

Copernicus Marine Service. (2025). Global ocean acidification – Mean sea water pH time series and trend from multi-observations reprocessing. \url{https://data.marine.copernicus.eu/product/GLOBAL_OMI_HEALTH_carbon_ph_area_averaged/description}

De'ath, G., Lough, J., and Fabricius, K. (2009). Declining coral calcification on the Great Barrier Reef. Science, 323, 116–119. https://doi.org/10.1126/science.1165283

Deryng, D., Conway, D., Ramankutty, N., Price, J., Warren, R., Jones, R., ... and Elliott, J. (2016). Regional disparities in the beneficial effects of rising CO$_2$ concentrations on crop water productivity. Nature Climate Change. https://doi.org/10.1038/nclimate2995

Gattuso, J. P., and Hansson, L. (Eds.). (2011). Ocean acidification: Background and history. Oxford University Press.

Gerhart, L. M., and Ward, J. K. (2010). Plant responses to low [CO$_2$] of the past. New Phytologist, 188, 674–695. https://nph.onlinelibrary.wiley.com/doi/pdf/10.1111/j.1469-8137.2010.03441.x

Haverd, V., B. Smith, J. G. Canadell, et al. (2020). Higher than expected CO$_2$ fertilization inferred from leaf to global observations. Global Change Biology, 26, 2390–2402. https://doi.org/10.1111/gcb.14950

Keenan, T. F., X. Luo, B. D. Stocker, et al. (2023). A constraint on historic growth in global photosynthesis due to rising CO2. Nature Climate Change 13(12): 1376-1381 DOI: 10.1038/s41558-023-01867-2.

Judd, E. J., Scotese, C. R., Young, S. A., et al. (2024). A 485-million-year history of Earth's surface temperature. Science, 385(6715). https://doi.org/10.1126/science.adk3705

Krissansen-Totton, J., Arney, G. N., and Catling, D. C. (2018). Constraining the climate and ocean pH of the early Earth with a geological carbon cycle model. Proceedings of the National Academy of Sciences, 115(6), 4105–4110. https://doi.org/10.1073/pnas.1721296115

Piao, S., X. Wang, T. Park, et al. (2020). Characteristics, drivers and feedbacks of global greening. Nature Reviews Earth \& Environment 1(1): 14-27 DOI: 10.1038/s43017-019-0001-x

Rae, J. W. B., Burke, A., Robinson, L. F., et al. (2018). CO$_2$ storage and release in the deep Southern Ocean on millennial to centennial timescales. Nature, 562, 569–573. https://doi.org/10.1038/s41586-018-0614-0

Revelle, R., and Fairbridge, R. W. (1957). Carbonate and carbon dioxide. In J. W. Hedgpeth (Ed.), Treatise on marine ecology and paleoecology (Vol. 1). Geological Society of America.

Ridd, P., Silva, E., and Stieglitz, T. (2013). Have coral calcification rates slowed in the last twenty years? Marine Geology, 346, 392–399. https://doi.org/10.1016/j.margeo.2013.09.002

Woods Hole Oceanographic Institution. (2023). Is the Great Barrier Reef making a comeback? https://www.whoi.edu/oceanus/feature/is-the-great-barrier-reef-making-a-comeback/

Zeng, Z., Piao, S. Li, L., et al. (2017). Climate mitigation from vegetation biophysical feedbacks during the past three decades. Nature Climate Change. https://doi.org/10.1038/nclimate3299

Zhang, Y., Liu, Y., Chen, X., et al. (2024). Less than 4% of dryland areas are projected to desertify despite increased aridity under climate change. Nature Communications Earth and Environment, 5. https://www.nature.com/articles/s43247-024-01463-y

Zhu, Z., Piao, S., Myneni, R. B., et al. (2016). Greening of the Earth and its drivers. Nature Climate Change, 6, 791–795. https://www.nature.com/articles/nclimate3004

\numberwithin{figure}{section}
\chapter{Menschliche Einflüsse auf das Klima}
\paragraph{Kapitelzusammenfassung}
\begin{quote}
Das globale Klima ist natürlicherweise auf allen Zeitskalen variabel. Anthropogene CO$_2$-Emissionen verstärken diese Variabilität, indem sie die gesamte Strahlungsenergiebilanz in der Atmosphäre verändern.

Das IPCC hat die Rolle der Sonne beim Klimawandel herabgespielt, aber es gibt plausible Rekonstruktionen der Sonneneinstrahlung, die darauf hindeuten, dass sie zur jüngsten Erwärmung beigetragen hat.

Klimaprojektionen basieren auf IPCC-Emissionsszenarien, die dazu tendierten, beobachtete Trends zu übertreffen. Die meisten akademischen Klimaauswirkungsstudien der letzten Jahre basieren auf dem extremen RCP 8.5-Szenario, das jetzt als unplausibel gilt; seine Verwendung als Business-as-usual-Szenario war irreführend.

Kohlenstoffkreislaufmodelle verbinden jährliche Emissionen mit dem Wachstum des atmosphärischen CO$_2$-Bestands. Während sich Modelle über die Rate der Land- und Ozean-CO$_2$-Aufnahme uneinig sind, stimmen alle darin überein, dass sie seit 1959 gestiegen ist.

Es gibt Belege dafür, dass Urbanisierungsverzerrungen in der Landerwärmungsaufzeichnung nicht vollständig aus Klimadatensätzen entfernt wurden.
\end{quote}

\section{Komponenten der Strahlungsantriebe und ihre Geschichte}
\subsection{Historischer Strahlungsantrieb}
Ein sich veränderndes Klima war die Norm während der gesamten 4.6 Milliarden Jahre langen Geschichte der Erde. Die Temperatur und Wettermuster der Erde verändern sich natürlich über Zeitskalen von Jahrzehnten bis zu Millionen von Jahren. Natürliche Variationen des Oberflächenklimas entstehen auf zwei Wegen. Interne Klimaschwankungen im Zusammenhang mit Zirkulationen in der Atmosphäre und den Ozeanen tauschen Energie, Wasser und Kohlenstoff zwischen der Atmosphäre, Ozeanen, Land und Eis aus. Externe Einflüsse auf das Klimasystem umfassen Variationen in der von der Sonne empfangenen Energie und die Auswirkungen von Vulkanausbrüchen. Menschliche Aktivitäten beeinflussen das Klima durch Veränderungen der Landnutzung und Landbedeckung. Menschen verändern auch die Zusammensetzung der Atmosphäre durch Emissionen von CO$_2$ und anderen Treibhausgasen und durch die Veränderung der Konzentration von Aerosolpartikeln in der Atmosphäre.

Die Erde wird durch das Sonnenlicht erwärmt, das sie absorbiert, und wird durch die Wärme gekühlt, die sie in den Weltraum abstrahlt. Über die Erdoberfläche gemittelt, beinhalten diese Prozesse jeweils Energieflüsse von etwa 240 Watt pro Quadratmeter (W/m²). Wenn sie im Gleichgewicht sind, gibt es keine äußeren Ursachen für Erwärmung oder Abkühlung. Sowohl menschliche als auch natürliche Einflüsse auf das Klima verändern dieses Gleichgewicht und verursachen damit Klimaveränderungen.

Einflüsse auf die Energiebilanz der Erde an der Spitze der Atmosphäre werden durch \emph{Strahlungsantrieb} quantifiziert, das Ausmaß, in dem sie das Erwärmungs-/Abkühlungsgleichgewicht stören; ein positiver Antrieb erwärmt, während ein negativer Antrieb kühlt. Die geschätzte Geschichte der wichtigsten Komponenten des Strahlungsantriebs seit 1750 durch das IPCC ist in den folgenden zwei Abbildungen aus seinem AR6 dargestellt.

\begin{figure}[H]
\begin{center}
\includegraphics[width=1.0\textwidth]{bilder/bilderKlima-0006.jpg}\\[1cm]
\end{center}
\caption{IPCC-Schätzungen der Komponenten des Strahlungsantriebs im Zeitverlauf. Die Schattierung zeigt
die Unsicherheitsbereiche an. Quelle: AR6 WGI Ch2 Abb. 10}
\end{figure}

\begin{figure}[H]
\begin{center}
\includegraphics[width=1.0\textwidth]{bilder/bilderKlima-0007.jpg}\\[1cm]
\end{center}
\caption{IPCC-Schätzungen der Veränderungen der Strahlungsantriebskomponenten von 1750 bis 2019. Quelle:
AR6 WGI Kap. 7 Abb. 7-6.}
\end{figure}

Diese Grafiken zeigen, dass der gesamte Strahlungsantrieb sowohl aus natürlichen als auch aus anthropogenen Komponenten besteht. Kohlendioxid ist der größte menschliche Einfluss auf das Klima und derjenige, der am relevantesten für den Einfluss der Nutzung fossiler Brennstoffe ist. Es übt einen erwärmenden Einfluss aus, indem es die Kühlkraft der Atmosphäre verringert. CO$_2$-Emissionen akkumulieren in der Atmosphäre, wie im folgenden Abschnitt beschrieben, sodass der erwärmende Einfluss wächst.

Andere anthropogene Strahlungsantriebe umfassen andere Treibhausgase (Methan, Lachgas, fluorierte Gase), Aerosole und Landnutzungsänderungen. Es gibt jedoch erhebliche Unsicherheiten in diesen Komponenten. Das IPCC identifiziert Wolken-Aerosol-Wechselwirkungen als die größte Unsicherheit im gesamten Strahlungsantrieb.

Eine wichtige natürliche Komponente ist die Sonneneinstrahlung. Variationen in der Sonnenaktivität sind gut dokumentiert und können das Klima beeinflussen. Das IPCC hat jedoch die Rolle solarer Variabilität beim Klimawandel heruntergespielt. Soon et al. (2021) stellten fest, dass die Konsensaussagen des IPCC zum solaren Antrieb vorzeitig durch die Unterdrückung abweichender wissenschaftlicher Meinungen formuliert wurden.

Eine weitere natürliche Strahlungsantriebskomponente sind vulkanische Aerosole, die einen episodischen kühlenden Einfluss ausüben. Box 4.1 im IPCC AR6-Bericht behandelt die Klimaauswirkungen von Vulkanausbrüchen und bemerkt drei explosive Vulkanausbrüche, die in der ersten Hälfte des 19. Jahrhunderts auftraten. Dazu gehörte der Tambora-Ausbruch von 1815, der zum 'Jahr ohne Sommer' führte, mit mehreren Ernteausfällen in der gesamten Nordhalbkugel. Es gibt Unsicherheit über das Vorzeichen der relativ kleinen Antriebskraft des Unterwasservulkans Hunga Tonga, der 2022 ausbrach (Jenkins et al. 2023, Schoeberl et al. 2024).

Abbildung 3.1.1 zeigt, dass die anthropogene Antriebskomponente vor etwa 1900 vernachlässigbar war und seitdem stetig gestiegen ist und heute auf fast \SI{3}{\watt\per\square\meter} angestiegen ist. Dies ist jedoch immer noch nur etwa 1 Prozent der ungestörten Strahlungsflüsse, was es zu einer Herausforderung macht, die Auswirkungen des anthropogenen Antriebs zu isolieren; modernste Satellitenschätzungen globaler Strahlungsenergieflusse sind nur auf wenige \si{\watt\per\square\meter} genau.

Natürliche Quellen globaler Energieungleichgewichte außer Vulkanen und der gesamten Sonneneinstrahlung (TSI) sind in diesen Grafiken nicht enthalten, da sie weitgehend unbekannt bleiben.

\subsection{Veränderung des atmosphärischen CO$_2$ seit 1958}

Der erwärmende Einfluss von Kohlendioxid hängt davon ab, wie viel \emph{zusätzliches} CO$_2$ sich in der Atmosphäre ansammelt – d.h. seine Konzentration über dem vorindustriellen Wert von \SI{280}{ppm}. Der CO$_2$-Gehalt, wie er am Mauna Loa-Observatorium in Hawaii aufgezeichnet wird, der allgemein als repräsentative globale Durchschnittskonzentration verwendet wird, ist online verfügbar unter \url{https://gml.noaa.gov/ccgg/trends/index.html}. Die Konzentration lag zu Beginn der Aufzeichnung 1959 bei etwa \SI{316}{ppm} und liegt jetzt bei etwa \SI{430}{ppm}, ein Anstieg von 36 Prozent. Am Ende der letzten Vereisung waren die CO$_2$-Konzentrationen auf etwa \SI{180}{ppm} gefallen. Wie in Kapitel 2 diskutiert, beginnen C3-Pflanzen bei CO$_2$-Konzentrationen unter etwa \SI{140}{ppm} zu sterben und C4-Pflanzen bei Konzentrationen unter \SI{100}{ppm}, sodass bei weiterem Fallen der CO$_2$-Konzentrationen das Pflanzenleben gefährdet gewesen wäre.


\begin{figure}[H]
\begin{center}
\includegraphics[width=1.0\textwidth]{bilder/bilderKlima-0009.png}\\[1cm]
\end{center}
\caption{Jährliche durchschnittliche CO2-Konzentrationen in der Atmosphäre (1959–2025) in ppm, gemessen am
Mauna Loa (blau). C3-Schwellenwert: Wert, unterhalb dessen C3-Pflanzen zu sterben beginnen (140 ppm, siehe
Kapitel 2). C4-Schwellenwert: Wert, unterhalb dessen C4-Pflanzen zu sterben beginnen (100 ppm, siehe Kapitel 2).
Glaziales Minimum: Mindestniveau während der letzten Eiszeiten (lila Pfeil). CO2-Datenquelle:
\url{https://gml.noaa.gov/ccgg/trends/index.html}}
\end{figure}

Der jährliche Konzentrationsanstieg ist nur etwa die Hälfte des emittierten CO$_2$, weil Land- und Ozeanprozesse derzeit \emph{überschüssiges} CO$_2$ mit einer Rate von etwa 50 Prozent der menschlichen Emissionen absorbieren. Zukünftige Konzentrationen und damit zukünftige menschliche Einflüsse auf das Klima hängen daher von zwei Komponenten ab: (1) zukünftige Raten globaler menschlicher CO$_2$-Emissionen und (2) wie schnell Land und Ozean zusätzliches CO$_2$ aus der Atmosphäre entfernen. Wir diskutieren jede dieser der Reihe nach.

\section{Zukünftige Emissionsszenarien und der Kohlenstoffkreislauf}

\subsection{Emissionsszenarien}

Die Bewertung der Gefahren zukünftiger THG-Emissionen erfordert Annahmen darüber, was diese Emissionen sein werden. Zukünftige Emissionen und damit menschliche Einflüsse auf das Klima werden von zukünftiger Demografie, Wirtschaftstätigkeit, Regulierung sowie Energie- und Agrartechnologien abhängen. Verschiedene Annahmen über jeden dieser Faktoren führen zu Projektionen von Treibhausgasemissionen und -konzentrationen, Aerosolkonzentrationen und Veränderungen der Landnutzung, die letztendlich zu Annahmen über anthropogenen Strahlungsantrieb kombiniert werden können.

Die großen Unsicherheiten über diese vielen Faktoren machen es unmöglich, zukünftige Emissionen präzise vorherzusagen. Stattdessen hat das IPCC verschiedene Szenario-Sets verwendet, die einen plausiblen Bereich von Möglichkeiten für Bevölkerung, Wirtschaft und Technologien abdecken sollen. Jüngste Versionen der Szenarien sind durch eine Zahl gekennzeichnet, die den anthropogenen Strahlungsantrieb angibt, der 2100 unter diesem Szenario erwartet wird. So entspricht ein mit "6" bezeichnetes Szenario \SI{6}{\watt\per\square\meter} menschlich induziertem Strahlungsantrieb (Erwärmung) am Ende des Jahrhunderts. (Erinnern Sie sich, der aktuelle anthropogene Strahlungsantrieb beträgt etwa \SI{2.7}{\watt\per\square\meter}.)

Obwohl das IPCC nicht behauptet, dass seine Emissionsszenarien Vorhersagen sind, werden sie oft als solche behandelt. Vergleiche vergangener Szenario-Gruppen mit Beobachtungen zeigen, dass IPCC-Emissionsprojektionen dazu tendierten, tatsächliche nachfolgende Emissionen zu überschätzen. Für den dritten und vierten IPCC-Bewertungsbericht wurde eine Reihe von Emissionsprojektionen aus dem Sonderbericht zu Emissionsszenarien verwendet; diese wurden als SRES-Szenarien bezeichnet. McKitrick et al. (2012) zeigten, dass die SRES-Szenario-Emissionsverteilung bei Umrechnung in Pro-Kopf-Werte im Vergleich zu beobachteten Trends nach oben verzerrt war. Die Verzerrung der SRES-Szenarien wurde durch die spätere Analyse von Hausfather et al. (2019) bestätigt, die zeigten, dass beobachtete atmosphärische CO$_2$-Konzentrationen dem unteren Ende des SRES-Bereichs und auch nachfolgender IPCC-Szenario-Bereiche folgten (Abbildung 3.2.1).

Für AR5 entwickelte das IPCC ein neues Set von Szenarien, die \emph{Representative Concentration Pathways} (RCPs) genannt wurden. Diese wurden durch eine Zahl identifiziert, die den Anstieg im Antrieb repräsentierte, und wurden daher RCP2.6, RCP4.5, RCP6.0 und RCP8.5 genannt. RCP2.6 (was einen anthropogenen Strahlungsantrieb 2100 von \SI{2.6}{\watt\per\square\meter} impliziert) beschreibt einen THG-Konzentrationspfad, der zu einer Erwärmung deutlich unter 2°C führt. Am anderen Ende der Skala ist RCP8.5 ein extremes Ergebnis, das fast \SI{5}{\celsius} Erwärmung von 1900 bis 2100 impliziert.

RCP8.5 kam dazu, als \emph{No-Policy-Baseline} oder \emph{Business-as-usual}-Szenario sowohl in der akademischen Literatur als auch in den populären Medien bezeichnet zu werden. Es wurde daher verwendet, um das Referenzergebnis zu generieren, das angeblich die Welt des 21. Jahrhunderts in Abwesenheit zunehmend strenger Emissionsreduktionspolitiken repräsentiert. Aber RCP8.5 war als ein Niedrig-Wahrscheinlichkeits-Hochemmissionsszenario gedacht, und seine Verwendung als Business-as-usual-Baseline wurde als grob irreführend kritisiert.\footnote{Dieses extreme Szenario ist nützlich für Modellierer, da ein großer Antrieb eine große Antwort (Erwärmung) generiert, was es einfacher macht, die Sensitivität eines Modells zu bewerten. Aber das ist sehr unterschiedlich davon, zu behaupten, es sei ein plausibles zukünftiges Ergebnis.} Hausfather und Peters (2020a), die in einem Kommentar in Nature schrieben, wiesen darauf hin, dass RCP8.5 als ein extremer Worst-Case entwickelt wurde, und sein Missbrauch als \emph{Business as usual}-Baseline hat zu einer großen Anzahl irreführender Studien und Medienberichterstattung geführt.

Die Unplausibilität des RCP8.5-Szenarios wurde von Burgess et al. (2021) untersucht. Die Unplausibilität von RCP8.5 sollte nicht als sehr unwahrscheinlich (z.B. 95. Perzentil) oder ein \emph{Worst Case} interpretiert werden, sondern eher als genuinen unplausibel aufgrund der Unplausibilität der Inputs, die erforderlich sind, um einen Antrieb von \SI{8.5}{\watt\per\square\meter} zu erreichen. Sie bemerkten, dass RCP8.5 bereits von beobachteten Trends in der Energienutzung abgewichen ist und die nahen zukünftigen Trends scharf von denen der Internationalen Energieagentur (IEA) abweichen, die marktbasierte Projektionen der Energienutzung für die kommenden Jahrzehnte bereitstellt. Pielke Jr. et al. (2022) zeigten weiter, dass die historischen und projizierten IEA-Trends nahe dem Boden der Umhüllungen sowohl der RCP-Projektionen als auch der jüngeren Shared Socioeconomic Pathway (SSP)-Szenario-Trends verlaufen.

Schwalm et al. (2020) verteidigten die Verwendung von RCP8.5 mit der Begründung, dass kumulative CO$_2$-Emissionen über 2005-2020 es enger verfolgen als die niedrigeren RCP-Szenarien. Sie argumentieren auch, dass eine modifizierte Version der IEA-Szenarien RCP8.5 in den kommenden Jahrzehnten eng verfolgt. Hausfather und Peters (2020b) antworteten, dass die Fertigkeit von RCP8.5 über diese 15 Jahre auf ausgleichende Fehler in seiner Repräsentation von CO$_2$ aus Brennstoffnutzung und Landnutzungsänderung zurückzuführen ist, und die scheinbare Übereinstimmung mit IEA in kommenden Jahrzehnten ist darauf zurückzuführen, dass Schwalm et al. sehr hohe Landnutzungsemissionen hinzufügten. Die eigenen projizierten CO$_2$-Emissionen der IEA verfolgen deutlich unter RCP8.5.

Weitverbreitete Verwendung von RCP8.5 als No-Policy-Baseline hat eine Verzerrung in Richtung Alarm in der Klimaauswirkungsliteratur geschaffen. Das Ausmaß dieses Problems wurde in einer Literaturanalyse von Pielke Jr. und Ritchie (2020) bestätigt. Sie fanden, dass etwa 16.800 wissenschaftliche Arbeiten, die zwischen 2010 und 2020 veröffentlicht wurden, das RCP8.5-Szenario verwendeten, wobei etwa 4.500 der Artikel RCP8.5 mit dem Konzept von \emph{Business-as-usual} verknüpften. Ihre Analyse zeigte, wie RCP8.5 nicht nur von einzelnen Forschern missbraucht wurde, sondern auch von einflussreichen Wissenschaftsagenturen wie dem IPCC und dem U.S. National Climate Assessment (USNCA), was direkt zu irreführender Berichterstattung in prominenten Medien geführt hat.

\begin{figure}[H]
\begin{center}
\includegraphics[width=1.0\textwidth]{bilder/bilderKlima-0011.png}\\[1cm]
\end{center}
\caption{Seit den 1970er Jahren haben aufeinanderfolgende Familien von Emissions- und Konzentrationsprognosen (farbige
Linien) die Beobachtungen (schwarze Linie) durchweg überschätzt. Quelle: Hausfather et al. (2019)
Abbildung S4.}
\end{figure}

Pielke und Ritchie (2020) berichteten, dass neue Studien, die RCP8.5 verwendeten, mit einer Rate von etwa 20 pro Tag veröffentlicht wurden, wobei etwa zwei pro Tag spezifisch RCP8.5 und \emph{business as usual} verknüpften. Sie schließen, dass die Klimaforschungsgemeinschaft ein Jahrzehnt damit verbracht hat, \emph{wissenschaftliche Ressourcen für Science Fiction zu verwenden} und dass \emph{Die wissenschaftliche Literatur ist in eine apokalyptische Richtung unausgewogen geworden.}

Das IPCC entwickelte ein neues Set von Szenarien für AR6, die \emph{Shared Socioeconomic Pathway} (SSP)-Szenarien, die die in den RCP- und SRES-Szenarien gezeigte Verzerrung fortgesetzt haben. Abbildung 3.2.2 zeigt die von der Internationalen Energieagentur (IEA) zusammengestellten gesamten globalen beobachteten CO$_2$-Emissionen, zusammengeführt mit der Emissionsprojektion der EIA unter Berücksichtigung von Energienutzungsprojektionen und aktuellen Politiken. Die anderen Linien zeigen den Bereich der IPCC SSP-Szenarien (SSP1-SSP5). Ab 2023 lagen globale CO$_2$-Emissionen deutlich unter SSP7.0 und waren sogar unter SSP2-4.5.

\begin{figure}[H]
\begin{center}
\includegraphics[width=1.0\textwidth]{bilder/bilderKlima-0012.jpg}\\[1cm]
\end{center}
\caption{Beobachtete und prognostizierte CO$_2$-Emissionen. Quelle: IPCC (SSP-Szenarien) und
Energy Information Administration (EIA). Grün: beobachtete historische Emissionen und EIA-Prognosen.
Andere Linien: SSP1-5. Datenquelle: Friedlingstein et al. (2024).}
\end{figure}

\subsection{Der Kohlenstoffkreislauf, der Emissionen und Konzentrationen in Beziehung setzt}
Kohlendioxidemissionen aus der Verbrennung fossiler Brennstoffe (und in geringerem Maße Entwaldung und Zementproduktion) haben zu stetig steigenden CO$_2$-Konzentrationen in der Atmosphäre geführt, wie in Abb. 3.1.3 gezeigt. Die Beziehung zwischen Emissionen und Konzentration wird durch den globalen Kohlenstoffkreislauf von Land- und Ozeanprozessen bestimmt, die Kohlenstoff mit der Atmosphäre austauschen. Unser Verständnis dieser Prozesse wurde von Crisp et al. (2021) überprüft.

Es gibt etwa \SI{850}{\giga\tonne} Kohlenstoff (\si{\giga\tonne\of{C}}) in der Erdatmosphäre\footnote{Weil CO$_2$ chemisch durch den Verlauf des Kohlenstoffkreislaufs transformiert wird, ist es zweckmäßiger, Kohlenstoffatome zu verfolgen statt CO$_2$-Moleküle. Eine Gigatonne Kohlenstoff (\si{\giga\tonne\of{C}}) entspricht etwa \SI{3.7}{\giga\tonne} CO$_2$.}, fast alles davon in der Form von CO$_2$. Jedes Jahr tauschen biologische Prozesse (Pflanzenwachstum und -zerfall) und physische Prozesse (Ozeanabsorption und -ausgasung) etwa \SI{200}{\giga\tonne\of{C}} dieses Kohlenstoffs mit der Erdoberfläche aus (ungefähr \SI{80}{\giga\tonne\of{C}} mit dem Land und \SI{120}{\giga\tonne\of{C}} mit den Ozeanen). Bevor menschliche Aktivitäten bedeutsam wurden, waren Entfernungen aus der Atmosphäre grob im Gleichgewicht mit Hinzufügungen. Aber die Verbrennung fossiler Brennstoffe (Kohle, Öl und Gas) entfernt Kohlenstoff aus dem Boden und fügt ihn dem jährlichen Austausch mit der Atmosphäre hinzu. Diese Hinzufügung (zusammen mit einem viel kleineren Beitrag aus der Zementherstellung) belief sich 2023 auf \SI{10.3}{\giga\tonne\of{C}} oder nur etwa 5 Prozent des jährlichen Austauschs mit der Atmosphäre.

Der Kohlenstoffkreislauf nimmt etwa 50 Prozent der kleinen jährlichen Injektion von Kohlenstoff der Menschheit in die Luft auf, indem er ihn natürlich durch Pflanzenwachstum und ozeanische Aufnahme sequestriert, während der Rest sich in der Atmosphäre ansammelt (Ciais et al., 2013). Aus diesem Grund beträgt der jährliche Anstieg der atmosphärischen CO$_2$-Konzentration im Durchschnitt nur etwa die Hälfte dessen, was naiv von menschlichen Emissionen erwartet würde.

Um zukünftige CO$_2$-Konzentrationen in der Atmosphäre und damit zukünftige menschliche Einflüsse auf das Klima zu projizieren, ist es wichtig zu wissen, wie sich der Kohlenstoffkreislauf in der Zukunft ändern könnte. Die historische Beinahe-Konstanz dieses 50-Prozent-Anteils bedeutet, dass je mehr CO$_2$ die Menschheit produziert hat, desto schneller hat die Natur es aus der Atmosphäre entfernt. Dieser 50-Prozent-Anteil ändert sich von Jahr zu Jahr etwas aufgrund natürlicher Kohlenstoffkreislauf-Ungleichgewichte durch El Ni\~no, La Ni\~na und variierende Wettermuster. Es gab auch eine erhebliche zusätzliche Reduktion des atmosphärischen CO$_2$ nach dem Ausbruch des Mount Pinatubo 1991, ein merkwürdiges Ergebnis, das noch erklärt werden muss (Angert et al., 2004).

Die Hauptprozesse, die überschüssiges CO$_2$ aus der Atmosphäre entfernen, sind verstärktes Wachstum der Landvegetation (besonders in hohen Breitengraden), eine gewisse Zunahme der Kohlenstoffsequestrierung in Böden und die Aufnahme von CO$_2$ durch den Ozean aufgrund des steigenden Partialdrucks von atmosphärischem CO$_2$ gegenüber dem in den Ozeanen gelösten CO$_2$. Alle zwanzig Landkohlenstoffkreislauf-Modelle, die vom Global Carbon Project verfolgt werden (Friedlingstein et al., 2024), zeigen, dass Landprozesse seit 1959 überschüssiges CO$_2$ mit steigender Rate entfernen. Dies stimmt mit einem \emph{globalen Ergrünungs}-Phänomen (Kapitel 2.1) überein, das von Satelliten seit Beginn der Überwachung der globalen Grünheit 1982 beobachtet wird.

Während Landvegetation positiv auf mehr atmosphärisches CO$_2$ reagiert hat, bleibt die Aufnahme von zusätzlichem CO$_2$ durch ozeanische biologische Prozesse zu ungewiss, um zuverlässig gemessen zu werden. Unser aktuelles Verständnis dieser und vieler weiterer Kohlenstoffkreislauf-Prozesse wurde von Crisp et al. (2021) überprüft.

\begin{figure}[H]
\begin{center}
\includegraphics[width=1.0\textwidth]{bilder/bilderKlima-0013.png}\\[1cm]
\end{center}
\caption{Trends der jährlichen CO2-Aufnahme (GtCO2 pro Jahr und Jahrzehnt) durch Landprozesse
im Zeitraum 1959–2023, simuliert durch 20 verschiedene dynamische globale Vegetationsmodelle, die regelmäßig
vom Global Carbon Project (Friedlingstein, 2024) veröffentlicht werden.}
\end{figure}

\subsubsection*{CO$_2$-Aufnahme durch Landprozesse}
Die Aufnahme von zusätzlichem CO$_2$ aus der Atmosphäre durch Landoberflächenprozesse (wie auch aus der globalen Ergrünung geschlossen) wurde mit 20 verschiedenen dynamischen globalen Vegetationsmodellen modelliert, deren Ausgaben jährlich vom Global Carbon Project aktualisiert werden (Friedlingstein, 2024). Wie in Abb. 3.2.3 zu sehen ist, stimmen alle diese Modelle darin überein, dass Vegetation und Böden Kohlenstoff aus der Atmosphäre sequestriert haben. Aber wir sehen auch, dass die langfristigen Trends über 1959 bis 2023 (65 Jahre) stark zwischen den Modellen variieren, um fast einen Faktor von 7. Dies zeigt, dass erhebliche Unsicherheit darüber besteht, wie schnell Landprozesse CO$_2$ aus der Atmosphäre entfernen, was wiederum Unsicherheit in zukünftigen atmosphärischen CO$_2$-Konzentrationen schafft, die dann Unsicherheit in Klimamodellsimulationen zukünftiger Klimaveränderungen erzeugen.

\subsubsection*{CO$_2$-Aufnahme durch Ozeanprozesse}
Die Aufnahme von zusätzlichem CO$_2$ aus der Atmosphäre durch Ozeanprozesse wurde mit 10 verschiedenen Ozeanbiogeochemie-Modellen modelliert, deren Ausgaben jährlich vom Global Carbon Project aktualisiert werden (Friedlingstein, 2024). Wie die Ergebnisse der Landmodelle stimmen alle Ozeanmodelle darin überein, dass die globalen Ozeane während 1959-2023 Kohlenstoff aus der Atmosphäre mit einer steigenden Rate sequestriert haben (Abb. 3.2.4). Im Gegensatz zu den Landmodellen zeigen die Ozeanmodelle jedoch eine viel bessere Übereinstimmung miteinander, wobei das Modell mit der schnellsten steigenden CO$_2$-Aufnahme nur 65 Prozent schneller ist als das Modell mit der langsamsten steigenden CO$_2$-Aufnahme. Trotz der relativen Übereinstimmung zwischen den Modellen bemerkt Friedlingstein et al. (2022), dass es erhebliche Diskrepanzen zwischen den verschiedenen Methoden über die Stärke der Ozeansenke im letzten Jahrzehnt gibt, besonders im Südozean.
Beachten Sie, dass der durchschnittliche Trend in der CO$_2$-Aufnahme über alle Landmodelle in Abb. 3.2.3 25 Prozent größer ist als der durchschnittliche Trend in der Ozeanaufnahme. Dies deutet darauf hin, dass Landprozesse in ihrer Fähigkeit, CO$_2$ zu entfernen, schneller zunehmen als Ozeanprozesse ihre CO$_2$-Sequestrierung steigern.

\begin{figure}[H]
\begin{center}
\includegraphics[width=1.0\textwidth]{bilder/bilderKlima-0015.png}\\[1cm]
\end{center}
\caption{Trends der jährlichen CO2-Aufnahme (GtCO2 pro Jahr und Jahrzehnt) durch Ozeanprozesse im Zeitraum
1959–2023, simuliert durch 10 verschiedene ozeanische Biogeochemie-Modelle, regelmäßig berichtet vom
Global Carbon Project (Friedlingstein, 2024).}
\end{figure}


\section{Urbanisierungseinfluss auf Temperaturtrends}
Historische Temperaturdaten über Land wurden hauptsächlich dort gesammelt, wo Menschen leben. Dies wirft das Problem auf, wie man nicht-klimatische Erwärmungssignale aufgrund von städtischen Wärmeinseln (UHI) und anderen Veränderungen der Landoberfläche herausfiltern kann. Wenn diese nicht entfernt werden, könnten die Daten beobachtete Erwärmung übermäßig Treibhausgasen zuschreiben. Das IPCC erkennt an, dass Rohtemperaturdaten mit UHI-Effekten kontaminiert sind, behauptet aber, Datenbereinigungsverfahren zu haben, die sie entfernen. Es ist eine offene Frage, ob diese Verfahren ausreichend sind.
AR6 spielte dieses Problem herunter, indem es sagte (WGI S. 235), dass keine neuen Beweise aufgetaucht seien, um die AR5-Feststellung zu ändern, dass Urbanisierung eine Aufwärtsverzerrung von nicht mehr als 10 Prozent im globalen Landerwärmungstrend verursacht. AR5 (WGI S. 189) zitierte ebenfalls die 10-Prozent-Obergrenze ohne Quellenangabe. AR4 (WGI S. 244) zitierte Jones et al. (1990) und Peterson et al. (1999) als Grundlage der Behauptung. Peterson et al. fanden keinen Unterschied in Trends zwischen ländlichen und städtischen Stichproben, obwohl ihre Definition von ländlich lokale Bevölkerungen bis zu 10.000 Personen einschloss, während der relative Einfluss der Urbanisierung deutlich darunter beginnt (Spencer et al., 2025). Jones et al. verglichen ländliche/städtische Erwärmung in drei Regionen: Ostaustralien, Ostchina und Westliche Sowjetunion. Ihre Definition von \emph{ländlich} umfasste Städte bis zu 10.000 in der ehemaligen Sowjetunion und bis zu 100.000 in China. Sie fanden relative Erwärmungsverzerrungen größer als 10 Prozent in diesen Gebieten, vermuteten aber, dass der Urbanisierungseffekt, gemittelt über die Gebiete, die sie nicht untersuchten, die globale Landverzerrung auf unter 10 Prozent des beobachteten Erwärmungstrends bringen würde.

Mehrere Arbeiten erschienen vor dem IPCC AR4, die argumentierten, dass der erwärmende Effekt von UHIs eine relativ große (30-50\%) Komponente zur beobachteten Erwärmung hinzufügte und nicht von Klimamodellen simuliert wurde (de Laat und Maurellis 2006, McKitrick und Michaels 2007). Diese Befunde basierten auf Korrelationen zwischen Orten maximaler Erwärmung über Land und Orten maximaler sozioökonomischer Entwicklung. AR4 behauptete (S. 244), dass diese Korrelationen ein Artefakt natürlicher atmosphärischer Zirkulationen waren und tatsächlich statistisch insignifikant, und verwarf die Befunde auf dieser Grundlage. Ihre Behauptung war kontrovers, weil sie ohne unterstützende Beweise präsentiert wurde. McKitrick (2010) und McKitrick und Nierenberg (2010) zeigten, dass die Berücksichtigung verschiedener vermuteter alternativer Erklärungen für die Korrelationen ihre Signifikanz nicht beeinflusste. AR5 (S. 189) räumte ein, dass AR4 \emph{keine expliziten Beweise} für seine Bewertung geliefert hatte und erkannte weiter auf der Grundlage dieser Arbeiten an, dass es \emph{signifikante Beweise für eine solche Kontamination der Aufzeichnung} gab, d.h. eine Erwärmungsverzerrung in der Landaufzeichnung. Wie bereits bemerkt, trugen sie jedoch an anderer Stelle im AR5-Bericht die AR4-Behauptung vor, dass es weniger als 10 Prozent der beobachteten Erwärmung seien. Außerdem gaben sie keine Warnung über die Verwendung der Landaufzeichnung für Klimamessungen, obwohl sie die Beweise für UHI-Kontamination einräumten. Kürzlich schätzten Soon et al. (2023) eine Urbanisierungsverzerrung in der nordhemisphärischen Landaufzeichnung über 1850-2018, die ausreicht, um den Trend in der gemischten Aufzeichnung von \SI{0.55}{\celsius} auf \SI{0.89}{\celsius} pro Jahrhundert zu erhöhen.

Einige Studien, die Beweise gegen UHI-Kontamination lieferten, verglichen Erwärmungsraten zwischen ländlichen und städtischen Standorten (Jones et al. 1990, Peterson et al. 1999, Wickham et al. 2013). Es ist nicht bekannt, ob solche Methoden UHI-Verzerrung erkennen könnten, selbst wenn sie vorhanden wäre. Der Einfluss der UHI-Erwärmung ist logarithmisch in der Bevölkerung, mit anderen Worten, er ist am stärksten bei niedriger Bevölkerungsdichte und flacht dann ab, wenn sich die lokale Urbanisierung ausdehnt (Oke 1973, Spencer et al. 2025). Daher beweist das Versagen, einen Unterschied in Erwärmungsraten zwischen städtischen und ländlichen Stationen zu finden, nicht die Abwesenheit von UHI-Kontamination. McKitrick (2013) lieferte eine empirische Demonstration, in der sich die ländlichen/städtischen Trends in einem Datensatz, der aus anderen Gründen als mit UHI-Verzerrung kontaminiert erwiesen wurde, nicht signifikant unterschieden.

Parker (2006) untersuchte eine Stichprobe städtischer Standorte und fand keinen Unterschied in Trends zwischen Untergruppen, die nach nächtlicher Windgeschwindigkeit unterteilt waren, und schloss auf dieser Grundlage, dass Urbanisierung kein signifikanter Faktor sein könnte. Auch hier ist die Frage, ob eine solche Methode UHI-Verzerrung finden würde, selbst wenn sie vorhanden wäre. McKitrick (2013) präsentierte ein Beispiel, in dem UHI-kontaminierte Daten keine signifikanten Trendunterschiede zeigten, wenn sie nach Windgeschwindigkeit stratifiziert wurden.

Die Herausforderung bei der Messung von UHI-Verzerrung besteht darin, lokale Temperaturveränderungen mit einer entsprechenden Veränderung in Bevölkerung oder Urbanisierung zu verknüpfen, anstatt mit einer statischen Klassifikationsvariable wie ländlich oder städtisch. Spencer et al. (2025) verwendeten neu verfügbare historische Bevölkerungsarchive, um eine solche Analyse durchzuführen, und fanden Beweise für signifikante UHI-Verzerrung in US-Sommertemperaturdaten.

Zusammenfassend, während es eindeutig Erwärmung in der Landaufzeichnung gibt, gibt es auch Beweise dafür, dass sie durch Urbanisierungsmuster nach oben verzerrt ist und dass diese Verzerrungen nicht vollständig durch die Datenverarbeitungsalgorithmen entfernt wurden, die zur Erstellung von Klimadatensätzen verwendet werden.

Literaturverzeichnis:

Angert, A., S. Biraud, Bonfils, C., Buermann, W. and I. Fung (2004). CO2 seasonality indicates origins of
post-Pinatubo sink. Geophysical Research Letters 31. \url{https://doi.org/10.1029/2004GL019760}

AR6: Intergovernmental Panel on Climate Change Sixth Assessment Report (2021) Working Group I
Contribution. www.ipcc.ch.

AR5: Intergovernmental Panel on Climate Change Fifth Assessment Report (2013) Working Group I
Contribution. www.ipcc.ch.

AR4: Intergovernmental Panel on Climate Change Fourth Assessment Report (2007) Working Group I
Contribution. www.ipcc.ch.

Burgess, Matthew et al (2021) Environmental Research Letters 16 014016
\url{}

Ciais, P., C. Sabine, G. Bala, L. Bopp, V. Brovkin,et al. (2013): Carbon and Other Biogeochemical Cycles.
In: Climate Change 2013: The Physical Science Basis. Contribution of Working Group I to the Fifth
Assessment Report of the Intergovernmental Panel on Climate Change [Stocker, T.F., D. Qin, G.-K.
Plattner, M. Tignor,et al. (eds.)]. Cambridge University Press, Cambridge, United Kingdom and New
York, NY, USA

Connolly, Roman, Willie Soon, Michael Connolly et al. (2021) How much has the Sun influenced
Northern Hemisphere temperature trends? An ongoing debate Research in Astronomy and
Astrophysics 21(6) doi: 10.1088/1674-4527/21/6/131 \url{https://iopscience.iop.org/article/10.1088/1674-
4527/21/6/131}

Crisp, David \& Dolman, Han (A.J.) \& Tanhua, Toste \& Mckinley, Galen \& Hauck, Judith \& Bastos, Ana
\& Sitch, Stephen \& Eggleston, Simon \& Aich, Valentin. (2022). How Well Do We Understand the
Land‐Ocean‐Atmosphere Carbon Cycle?. Reviews of Geophysics. 60. 10.1029/2021RG000736.

De Laat, A.T.J., and A.N. Maurellis (2006), Evidence for influence of anthropogenic surface processes on
lower tropospheric and surface temperature trends, International Journal of Climatology 26:897—913.

Friedlingstein, P., and 95 co-authors (2024): Global Carbon Budget 2024, Earth System Science Data
14(4), https://essd.copernicus.org/preprints/essd-2024-519

Hausfather et al. (2019) “Evaluating the Performance of Past Climate Model Projections” Geophysical
Research Letters 47(1) https://doi.org/10.1029/2019GL085378

Hausfather, Z. and G. Peters (2020a) “Emissions – the ‘business as usual’ story is misleading” Nature 29
January 2020 https://www.nature.com/articles/d41586-020-00177-3

Hausfather, Z. and G. Peters (2020b) RCP8.5 is a problematic scenario for near-term emissions.
Proceedings of the National. Academy of Sciences 117, 27791–27792 (2020)

Jenkins, S., Smith, C., Allen, M. et al. Tonga eruption increases chance of temporary surface temperature
anomaly above 1.5 °C. Nature Climate Change. 13, 127–129 (2023). https://doi.org/10.1038/s41558-
022-01568-2

Jones, P. D., P. Y. Groisman, M. Coughlan, N. Plummer, W.-C. Wang, and T. R. Karl (1990),
Assessment of urbanization effects in time series of surface air temperature over land, Nature, 347,
169 – 172

Liu, Pengfei et al. (2021) “Improved estimates of preindustrial biomass burning reduce the magnitude of
aerosol climate forcing in the Southern Hemisphere” Science Advances 7(22) May 2021
https://doi.org/10.1126/sciadv.abc1379

McKitrick, R.R. and P.J. Michaels (2007), Quantifying the influence of anthropogenic surface processes
and inhomogeneities on gridded global climate data, Journal of Geophysical Research, 112, D24S09,
doi:10.1029/2007JD008465.

McKitrick, Ross R. (2010) Atmospheric Oscillations Do Not Explain the Temperature-Industrialization
Correlation. Statistics Politics and Policy Vol 1. No. 1., July 2010

McKitrick, Ross R. (2013) Encompassing Tests of Socioeconomic Signals in Surface Climate
Data. Climatic Change doi 10.1007/s10584-013-0793-5. Volume 120, Issue 1-2.
http://link.springer.com/article/10.1007%2Fs10584-013-0793-5

McKitrick, Ross R. and Nicolas Nierenberg (2010) Socioeconomic Patterns in Climate Data. Journal of
Economic and Social Measurement, 35(3,4) pp. 149-175. DOI 10.3233/JEM-2010-0336

McKitrick, Ross R., Mark Strazicich and Junsoo Lee (2012) “Long-Term Forecasting of Global Carbon
Dioxide Emissions: Reducing Uncertainties Using a Per-Capita Approach.” Journal of
Forecasting, Vol 32, Issue 5, pp 435-451 DOI: 10.1002/for.2248.

Oke, T.R., 1973: City size and the urban heat island, Atmospheric Environment 7, 769-77922

Parker, D.E. (2006) “A Demonstration that Large-Scale Warming is not Urban.” Journal of Climate
19:2882—2895.

Peterson, Thomas C., Kevin P. Gallo, Jay Lawrimore, Timothy W. Owen, Alex Huang, David A.
McKittrick (1999) Global rural temperature trends. Geophysical Research Letters February 1999
https://doi.org/10.1029/1998GL900322

Pielke Jr., Roger and Ritchie, Justin (2020) “Systemic Misuse of Scenarios in Climate Research and
Assessment” Social Sciences Research Network April 2020, available at:
https://ssrn.com/abstract=3581777

Pielke Jr, R., Burgess, M. G., \& Ritchie, J. (2022). Plausible 2005-2050 emissions scenarios project
between 2 and 3 degrees C of warming by 2100. Environmental Research Letters 17 024027
https://iopscience.iop.org/article/10.1088/1748-9326/ac4ebf/pdf

Scaffeta, Nicola, Richard C. Willson, Jae N. Lee and Dong Wu (2019) Modeling Quiet Solar Luminosity
Variability from TSI Satellite Measurements and Proxy Models during 1980–2018. Remote Sensing
11(21) 2569 https://doi.org/10.3390/rs11212569

Schoeberl, M.R., Y. Wang, G. Taha, D.J. Zawada, R. Ueyama and A. Dessler, 2024. Evolution of the
climate forcing during the two years after the Hunga Tonga-Hunga Ha’apai eruption. Journal of
Geophysical Research., 129.

Schwalm, C.R., S. Glendon, P. B. Duffy (2020) RCP8.5 tracks cumulative CO2 emissions. Proceedings of
the National Academy of Sciences U.S.A. 117, 19656–19657 (2020).

Soon,W.; Connolly, R.; Connolly, M.; Akasofu, S.-I.; Baliunas, S.; et al. (2023) The Detection and
Attribution of Northern Hemisphere Land Surface Warming (1850–2018) in Terms of Human and
Natural Factors: Challenges of Inadequate Data. Climate 2023, 11, 179.
https://doi.org/10.3390/cli11090179

Spencer, Roy W, John R Christy and William D. Braswell (2025) Urban Heat Island Effects in U.S.
Summer Surface Temperature Data, 1895–2023 Journal of Applied Meteorology and Climatology
April 2025 https://doi.org/10.1175/JAMC-D-23-0199.1

Wickham C, R Rohde , RA Muller, J Wurtele, J Curry, et al. (2013) Influence of Urban Heating on the
Global Temperature Land Average using Rural Sites Identified from MODIS Classifications.
Geoinformatics and Geostatistics: An Overview 1:2.

Zacharias, Pia (2014) An Independent Review of Existing Total Solar Irradiance Records. Surveys in
Geophysics 35 pp. 897—912 https://link.springer.com/article/10.1007/s10712-014-9294-y

\cleardoublepage
\chapter*{TEIL II: KLIMAREAKTION AUF CO$_2$-EMISSIONEN}
\addcontentsline{toc}{chapter}{TEIL II: KLIMAREAKTION AUF CO$_2$-EMISSIONEN}
\cleardoublepage
\chapter{Klimasensitivität bezüglich CO$_2$-Einwirkung}
\paragraph{Kapitelzusammenfassung}
\begin{quote}
TODO
\end{quote}
% --- Literatur ---
% \begin{thebibliography}{9}
% \bibitem{key} Autor. \emph{Titel}. Verlag/Journal, Jahr.
% \end{thebibliography}

\end{document}